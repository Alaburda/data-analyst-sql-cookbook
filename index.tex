% Options for packages loaded elsewhere
\PassOptionsToPackage{unicode}{hyperref}
\PassOptionsToPackage{hyphens}{url}
\PassOptionsToPackage{dvipsnames,svgnames,x11names}{xcolor}
%
\documentclass[
  letterpaper,
  DIV=11,
  numbers=noendperiod]{scrreprt}

\usepackage{amsmath,amssymb}
\usepackage{iftex}
\ifPDFTeX
  \usepackage[T1]{fontenc}
  \usepackage[utf8]{inputenc}
  \usepackage{textcomp} % provide euro and other symbols
\else % if luatex or xetex
  \usepackage{unicode-math}
  \defaultfontfeatures{Scale=MatchLowercase}
  \defaultfontfeatures[\rmfamily]{Ligatures=TeX,Scale=1}
\fi
\usepackage{lmodern}
\ifPDFTeX\else  
    % xetex/luatex font selection
\fi
% Use upquote if available, for straight quotes in verbatim environments
\IfFileExists{upquote.sty}{\usepackage{upquote}}{}
\IfFileExists{microtype.sty}{% use microtype if available
  \usepackage[]{microtype}
  \UseMicrotypeSet[protrusion]{basicmath} % disable protrusion for tt fonts
}{}
\makeatletter
\@ifundefined{KOMAClassName}{% if non-KOMA class
  \IfFileExists{parskip.sty}{%
    \usepackage{parskip}
  }{% else
    \setlength{\parindent}{0pt}
    \setlength{\parskip}{6pt plus 2pt minus 1pt}}
}{% if KOMA class
  \KOMAoptions{parskip=half}}
\makeatother
\usepackage{xcolor}
\setlength{\emergencystretch}{3em} % prevent overfull lines
\setcounter{secnumdepth}{5}
% Make \paragraph and \subparagraph free-standing
\ifx\paragraph\undefined\else
  \let\oldparagraph\paragraph
  \renewcommand{\paragraph}[1]{\oldparagraph{#1}\mbox{}}
\fi
\ifx\subparagraph\undefined\else
  \let\oldsubparagraph\subparagraph
  \renewcommand{\subparagraph}[1]{\oldsubparagraph{#1}\mbox{}}
\fi

\usepackage{color}
\usepackage{fancyvrb}
\newcommand{\VerbBar}{|}
\newcommand{\VERB}{\Verb[commandchars=\\\{\}]}
\DefineVerbatimEnvironment{Highlighting}{Verbatim}{commandchars=\\\{\}}
% Add ',fontsize=\small' for more characters per line
\usepackage{framed}
\definecolor{shadecolor}{RGB}{241,243,245}
\newenvironment{Shaded}{\begin{snugshade}}{\end{snugshade}}
\newcommand{\AlertTok}[1]{\textcolor[rgb]{0.68,0.00,0.00}{#1}}
\newcommand{\AnnotationTok}[1]{\textcolor[rgb]{0.37,0.37,0.37}{#1}}
\newcommand{\AttributeTok}[1]{\textcolor[rgb]{0.40,0.45,0.13}{#1}}
\newcommand{\BaseNTok}[1]{\textcolor[rgb]{0.68,0.00,0.00}{#1}}
\newcommand{\BuiltInTok}[1]{\textcolor[rgb]{0.00,0.23,0.31}{#1}}
\newcommand{\CharTok}[1]{\textcolor[rgb]{0.13,0.47,0.30}{#1}}
\newcommand{\CommentTok}[1]{\textcolor[rgb]{0.37,0.37,0.37}{#1}}
\newcommand{\CommentVarTok}[1]{\textcolor[rgb]{0.37,0.37,0.37}{\textit{#1}}}
\newcommand{\ConstantTok}[1]{\textcolor[rgb]{0.56,0.35,0.01}{#1}}
\newcommand{\ControlFlowTok}[1]{\textcolor[rgb]{0.00,0.23,0.31}{#1}}
\newcommand{\DataTypeTok}[1]{\textcolor[rgb]{0.68,0.00,0.00}{#1}}
\newcommand{\DecValTok}[1]{\textcolor[rgb]{0.68,0.00,0.00}{#1}}
\newcommand{\DocumentationTok}[1]{\textcolor[rgb]{0.37,0.37,0.37}{\textit{#1}}}
\newcommand{\ErrorTok}[1]{\textcolor[rgb]{0.68,0.00,0.00}{#1}}
\newcommand{\ExtensionTok}[1]{\textcolor[rgb]{0.00,0.23,0.31}{#1}}
\newcommand{\FloatTok}[1]{\textcolor[rgb]{0.68,0.00,0.00}{#1}}
\newcommand{\FunctionTok}[1]{\textcolor[rgb]{0.28,0.35,0.67}{#1}}
\newcommand{\ImportTok}[1]{\textcolor[rgb]{0.00,0.46,0.62}{#1}}
\newcommand{\InformationTok}[1]{\textcolor[rgb]{0.37,0.37,0.37}{#1}}
\newcommand{\KeywordTok}[1]{\textcolor[rgb]{0.00,0.23,0.31}{#1}}
\newcommand{\NormalTok}[1]{\textcolor[rgb]{0.00,0.23,0.31}{#1}}
\newcommand{\OperatorTok}[1]{\textcolor[rgb]{0.37,0.37,0.37}{#1}}
\newcommand{\OtherTok}[1]{\textcolor[rgb]{0.00,0.23,0.31}{#1}}
\newcommand{\PreprocessorTok}[1]{\textcolor[rgb]{0.68,0.00,0.00}{#1}}
\newcommand{\RegionMarkerTok}[1]{\textcolor[rgb]{0.00,0.23,0.31}{#1}}
\newcommand{\SpecialCharTok}[1]{\textcolor[rgb]{0.37,0.37,0.37}{#1}}
\newcommand{\SpecialStringTok}[1]{\textcolor[rgb]{0.13,0.47,0.30}{#1}}
\newcommand{\StringTok}[1]{\textcolor[rgb]{0.13,0.47,0.30}{#1}}
\newcommand{\VariableTok}[1]{\textcolor[rgb]{0.07,0.07,0.07}{#1}}
\newcommand{\VerbatimStringTok}[1]{\textcolor[rgb]{0.13,0.47,0.30}{#1}}
\newcommand{\WarningTok}[1]{\textcolor[rgb]{0.37,0.37,0.37}{\textit{#1}}}

\providecommand{\tightlist}{%
  \setlength{\itemsep}{0pt}\setlength{\parskip}{0pt}}\usepackage{longtable,booktabs,array}
\usepackage{calc} % for calculating minipage widths
% Correct order of tables after \paragraph or \subparagraph
\usepackage{etoolbox}
\makeatletter
\patchcmd\longtable{\par}{\if@noskipsec\mbox{}\fi\par}{}{}
\makeatother
% Allow footnotes in longtable head/foot
\IfFileExists{footnotehyper.sty}{\usepackage{footnotehyper}}{\usepackage{footnote}}
\makesavenoteenv{longtable}
\usepackage{graphicx}
\makeatletter
\def\maxwidth{\ifdim\Gin@nat@width>\linewidth\linewidth\else\Gin@nat@width\fi}
\def\maxheight{\ifdim\Gin@nat@height>\textheight\textheight\else\Gin@nat@height\fi}
\makeatother
% Scale images if necessary, so that they will not overflow the page
% margins by default, and it is still possible to overwrite the defaults
% using explicit options in \includegraphics[width, height, ...]{}
\setkeys{Gin}{width=\maxwidth,height=\maxheight,keepaspectratio}
% Set default figure placement to htbp
\makeatletter
\def\fps@figure{htbp}
\makeatother
\newlength{\cslhangindent}
\setlength{\cslhangindent}{1.5em}
\newlength{\csllabelwidth}
\setlength{\csllabelwidth}{3em}
\newlength{\cslentryspacingunit} % times entry-spacing
\setlength{\cslentryspacingunit}{\parskip}
\newenvironment{CSLReferences}[2] % #1 hanging-ident, #2 entry spacing
 {% don't indent paragraphs
  \setlength{\parindent}{0pt}
  % turn on hanging indent if param 1 is 1
  \ifodd #1
  \let\oldpar\par
  \def\par{\hangindent=\cslhangindent\oldpar}
  \fi
  % set entry spacing
  \setlength{\parskip}{#2\cslentryspacingunit}
 }%
 {}
\usepackage{calc}
\newcommand{\CSLBlock}[1]{#1\hfill\break}
\newcommand{\CSLLeftMargin}[1]{\parbox[t]{\csllabelwidth}{#1}}
\newcommand{\CSLRightInline}[1]{\parbox[t]{\linewidth - \csllabelwidth}{#1}\break}
\newcommand{\CSLIndent}[1]{\hspace{\cslhangindent}#1}

\KOMAoption{captions}{tableheading}
\makeatletter
\makeatother
\makeatletter
\@ifpackageloaded{bookmark}{}{\usepackage{bookmark}}
\makeatother
\makeatletter
\@ifpackageloaded{caption}{}{\usepackage{caption}}
\AtBeginDocument{%
\ifdefined\contentsname
  \renewcommand*\contentsname{Table of contents}
\else
  \newcommand\contentsname{Table of contents}
\fi
\ifdefined\listfigurename
  \renewcommand*\listfigurename{List of Figures}
\else
  \newcommand\listfigurename{List of Figures}
\fi
\ifdefined\listtablename
  \renewcommand*\listtablename{List of Tables}
\else
  \newcommand\listtablename{List of Tables}
\fi
\ifdefined\figurename
  \renewcommand*\figurename{Figure}
\else
  \newcommand\figurename{Figure}
\fi
\ifdefined\tablename
  \renewcommand*\tablename{Table}
\else
  \newcommand\tablename{Table}
\fi
}
\@ifpackageloaded{float}{}{\usepackage{float}}
\floatstyle{ruled}
\@ifundefined{c@chapter}{\newfloat{codelisting}{h}{lop}}{\newfloat{codelisting}{h}{lop}[chapter]}
\floatname{codelisting}{Listing}
\newcommand*\listoflistings{\listof{codelisting}{List of Listings}}
\makeatother
\makeatletter
\@ifpackageloaded{caption}{}{\usepackage{caption}}
\@ifpackageloaded{subcaption}{}{\usepackage{subcaption}}
\makeatother
\makeatletter
\@ifpackageloaded{tcolorbox}{}{\usepackage[skins,breakable]{tcolorbox}}
\makeatother
\makeatletter
\@ifundefined{shadecolor}{\definecolor{shadecolor}{rgb}{.97, .97, .97}}
\makeatother
\makeatletter
\makeatother
\makeatletter
\makeatother
\ifLuaTeX
  \usepackage{selnolig}  % disable illegal ligatures
\fi
\IfFileExists{bookmark.sty}{\usepackage{bookmark}}{\usepackage{hyperref}}
\IfFileExists{xurl.sty}{\usepackage{xurl}}{} % add URL line breaks if available
\urlstyle{same} % disable monospaced font for URLs
\hypersetup{
  pdftitle={data-analyst-sql-cookbook},
  pdfauthor={Norah Jones},
  colorlinks=true,
  linkcolor={blue},
  filecolor={Maroon},
  citecolor={Blue},
  urlcolor={Blue},
  pdfcreator={LaTeX via pandoc}}

\title{data-analyst-sql-cookbook}
\author{Norah Jones}
\date{2023-10-18}

\begin{document}
\maketitle
\ifdefined\Shaded\renewenvironment{Shaded}{\begin{tcolorbox}[interior hidden, sharp corners, boxrule=0pt, borderline west={3pt}{0pt}{shadecolor}, enhanced, breakable, frame hidden]}{\end{tcolorbox}}\fi

\renewcommand*\contentsname{Table of contents}
{
\hypersetup{linkcolor=}
\setcounter{tocdepth}{2}
\tableofcontents
}
\bookmarksetup{startatroot}

\hypertarget{preface}{%
\chapter*{Preface}\label{preface}}
\addcontentsline{toc}{chapter}{Preface}

\markboth{Preface}{Preface}

I have a terrible memory for things that I use rarely. When it comes to
SQL, there are certain code patterns that are extremely useful for
setting up a data model but are basically used once. During the past few
years of writing this book, I have also collected some useful tidbits of
information that are also VERY valuable to someone starting out as a
data analyst. Therefore, this book is called the Data Analyst's SQL
Cookbook. I wrote this so that I have all of my useful patterns in one
place and so that I can look them up and share them with others.

This book is also open-sourced - your input is most welcome! If you have
a useful SQL pattern that you would like to share, please submit a pull
request to the GitHub repository. If you have a question or comment,
please submit an issue to the GitHub repository.

This book was born out of a need to collect SQL queries that were too
infrequently used to be memorized but too frequently used to be looked
up. The queries are organized by topic and are meant to be used as a
reference. The book is also meant to be a living document that will be
updated as new queries are discovered and old queries are improved.

Furthermore, these SQL patterns are not universally known! I have found
that many developers are not aware of these patterns and are reinventing
the wheel when they encounter these problems. I hope that this book will
help to spread these patterns and make them more widely known. My team
has been victim to the same problem - we found a solution on
StackOverflow only to find a better solution a year down the line.

\bookmarksetup{startatroot}

\hypertarget{introduction}{%
\chapter{Introduction}\label{introduction}}

\bookmarksetup{startatroot}

\hypertarget{introduction-1}{%
\chapter{Introduction}\label{introduction-1}}

This is a book created from markdown and executable code.

See Knuth (1984) for additional discussion of literate programming.

\begin{Shaded}
\begin{Highlighting}[]

\KeywordTok{select} \OperatorTok{*} \KeywordTok{from}\NormalTok{ playlists;}
\end{Highlighting}
\end{Shaded}

\begin{Shaded}
\begin{Highlighting}[]
\DecValTok{1} \SpecialCharTok{+} \DecValTok{1}
\end{Highlighting}
\end{Shaded}

\begin{verbatim}
[1] 2
\end{verbatim}

\bookmarksetup{startatroot}

\hypertarget{untitled}{%
\chapter{Untitled}\label{untitled}}

\bookmarksetup{startatroot}

\hypertarget{intersecting-dates}{%
\chapter{Intersecting Dates}\label{intersecting-dates}}

Let's say you have a table of subscribtion that all different start and
end dates. How would you filter down a list of subscriptions to show
those that were active within a time range? In other words, how do you
find rows that have intersecting dates?

For example, here are all subscribers that had active subscriptions in
2023:

\begin{Shaded}
\begin{Highlighting}[]

\KeywordTok{select} \OperatorTok{*}
\KeywordTok{from}\NormalTok{ subscribers}
\KeywordTok{where}\NormalTok{ date\_from }\OperatorTok{\textless{}=} \StringTok{\textquotesingle{}2023{-}12{-}31\textquotesingle{}}
\KeywordTok{and}\NormalTok{ date\_to }\OperatorTok{\textgreater{}=} \StringTok{\textquotesingle{}2023{-}01{-}01\textquotesingle{}}
\end{Highlighting}
\end{Shaded}

\bookmarksetup{startatroot}

\hypertarget{calculating-date-ranges-based-on-gaps}{%
\chapter{Calculating date ranges based on
gaps}\label{calculating-date-ranges-based-on-gaps}}

Let's say we have subscriptions but we need to show a start date and an
end date of gaps between subscriptions. For example, if I subscribed
from 2023-01-01 to 2023-05-31 and then from 2023-07-01 to 2023-12-31, I
would want to return a row that said I was not a subscriber from
2023-06-01 to 2023-06-30.

\begin{Shaded}
\begin{Highlighting}[]
\KeywordTok{SELECT}   
\NormalTok{  seqval }\OperatorTok{+} \DecValTok{1} \KeywordTok{AS}\NormalTok{ start\_range,   }
\NormalTok{  (}
    \KeywordTok{SELECT} 
      \FunctionTok{MIN}\NormalTok{(B.seqval)    }
    \KeywordTok{FROM}\NormalTok{ dbo.NumSeq }\KeywordTok{AS}\NormalTok{ B    }
    \KeywordTok{WHERE}\NormalTok{ B.seqval }\OperatorTok{\textgreater{}}\NormalTok{ A.seqval}
\NormalTok{    ) }\OperatorTok{{-}} \DecValTok{1} \KeywordTok{AS}\NormalTok{ end\_range }
\KeywordTok{FROM}\NormalTok{ dbo.NumSeq }\KeywordTok{AS}\NormalTok{ A }
\KeywordTok{WHERE} \KeywordTok{NOT} \KeywordTok{EXISTS}\NormalTok{ (}
  \KeywordTok{SELECT} \OperatorTok{*} \KeywordTok{FROM}\NormalTok{ dbo.NumSeq }\KeywordTok{AS}\NormalTok{ B    }
  \KeywordTok{WHERE}\NormalTok{ B.seqval }\OperatorTok{=}\NormalTok{ A.seqval }\OperatorTok{+} \DecValTok{1}\NormalTok{)}
\KeywordTok{AND}\NormalTok{ seqval }\OperatorTok{\textless{}}\NormalTok{ (}\KeywordTok{SELECT} \FunctionTok{MAX}\NormalTok{(seqval) }\KeywordTok{FROM}\NormalTok{ dbo.NumSeq);}
\end{Highlighting}
\end{Shaded}

This solution is based on subqueries. In order to understand it you
should first focus on the filtering activity in the WHERE clause and
then proceed to the activity in the SELECT list. The purpose of the NOT
EXISTS predicate in the WHERE clause is to filter only points that are a
point before a gap. You can identify a point before a gap when you see
that for such a point, the value plus 1 doesn't exist in the sequence.
The purpose of the second predicate in the WHERE clause is to filter out
the maximum value from the sequence because it represents the point
before infinity, which does not concern us.

\bookmarksetup{startatroot}

\hypertarget{sessionization}{%
\chapter{Sessionization}\label{sessionization}}

\bookmarksetup{startatroot}

\hypertarget{islands-problem}{%
\chapter{Islands Problem}\label{islands-problem}}

\begin{Shaded}
\begin{Highlighting}[]

\KeywordTok{SELECT} 
  \FunctionTok{MIN}\NormalTok{(seqval) }\KeywordTok{AS}\NormalTok{ start\_range, }
  \FunctionTok{MAX}\NormalTok{(seqval) }\KeywordTok{AS}\NormalTok{ end\_range }
\KeywordTok{FROM}\NormalTok{ (}
  \KeywordTok{SELECT} 
\NormalTok{    seqval, }
\NormalTok{    seqval }\OperatorTok{{-}} \FunctionTok{ROW\_NUMBER}\NormalTok{() }\KeywordTok{OVER}\NormalTok{ (}\KeywordTok{ORDER} \KeywordTok{BY}\NormalTok{ seqval) }\KeywordTok{AS}\NormalTok{ grp}
  \KeywordTok{FROM}\NormalTok{ dbo.NumSeq}
\NormalTok{  ) }\KeywordTok{AS}\NormalTok{ D }\KeywordTok{GROUP} \KeywordTok{BY}\NormalTok{ grp;}
\end{Highlighting}
\end{Shaded}

\bookmarksetup{startatroot}

\hypertarget{section}{%
\chapter{}\label{section}}

\bookmarksetup{startatroot}

\hypertarget{summary}{%
\chapter{Summary}\label{summary}}

In summary, this book has no content whatsoever.

\begin{Shaded}
\begin{Highlighting}[]
\DecValTok{1} \SpecialCharTok{+} \DecValTok{1}
\end{Highlighting}
\end{Shaded}

\begin{verbatim}
[1] 2
\end{verbatim}

\bookmarksetup{startatroot}

\hypertarget{references}{%
\chapter*{References}\label{references}}
\addcontentsline{toc}{chapter}{References}

\markboth{References}{References}

\hypertarget{refs}{}
\begin{CSLReferences}{1}{0}
\leavevmode\vadjust pre{\hypertarget{ref-knuth84}{}}%
Knuth, Donald E. 1984. {``Literate Programming.''} \emph{Comput. J.} 27
(2): 97--111. \url{https://doi.org/10.1093/comjnl/27.2.97}.

\end{CSLReferences}

\bookmarksetup{startatroot}

\hypertarget{pivoting-and-unpivoting}{%
\chapter{Pivoting and Unpivoting}\label{pivoting-and-unpivoting}}

At this point it's likely you're using a database that supports pivoting
and unpivoting but it's good to know how to do it yourself.

\bookmarksetup{startatroot}

\hypertarget{pivoting}{%
\chapter{Pivoting}\label{pivoting}}

The most basic way to pivot is to use a CASE statement for each column
you want to pivot.

\begin{Shaded}
\begin{Highlighting}[]

\KeywordTok{select} 
\NormalTok{  ts\_id,}
  \FunctionTok{sum}\NormalTok{(}\ControlFlowTok{case} \ControlFlowTok{when}\NormalTok{ ts\_year }\OperatorTok{=} \DecValTok{2020} \ControlFlowTok{then}\NormalTok{ ts\_value }\ControlFlowTok{end}\NormalTok{) }\KeywordTok{as}\NormalTok{ ts\_2020,}
  \FunctionTok{sum}\NormalTok{(}\ControlFlowTok{case} \ControlFlowTok{when}\NormalTok{ ts\_year }\OperatorTok{=} \DecValTok{2021} \ControlFlowTok{then}\NormalTok{ ts\_value }\ControlFlowTok{end}\NormalTok{) }\KeywordTok{as}\NormalTok{ ts\_2021}
\KeywordTok{from}\NormalTok{ yearly\_values\_long}
\KeywordTok{group} \KeywordTok{by}\NormalTok{ ts\_id}
\end{Highlighting}
\end{Shaded}

\bookmarksetup{startatroot}

\hypertarget{advanced-pivoting}{%
\chapter{Advanced Pivoting}\label{advanced-pivoting}}

Why I like pivoting in SQL is that I can create arbitrary case when
statements to control how my data is pivoted:

\begin{Shaded}
\begin{Highlighting}[]

\KeywordTok{select} 
\NormalTok{  ts\_id,}
  \FunctionTok{sum}\NormalTok{(}\ControlFlowTok{case} \ControlFlowTok{when}\NormalTok{ ts\_year }\OperatorTok{=} \DecValTok{2020} \KeywordTok{and}\NormalTok{ ts\_value }\OperatorTok{\textgreater{}} \FloatTok{0.5} \ControlFlowTok{then}\NormalTok{ ts\_value }\ControlFlowTok{end}\NormalTok{) }\KeywordTok{as}\NormalTok{ ts\_2020,}
  \FunctionTok{sum}\NormalTok{(}\ControlFlowTok{case} \ControlFlowTok{when}\NormalTok{ ts\_year }\OperatorTok{=} \DecValTok{2021} \KeywordTok{and}\NormalTok{ ts\_value }\OperatorTok{\textgreater{}} \FloatTok{0.5} \ControlFlowTok{then}\NormalTok{ ts\_value }\ControlFlowTok{end}\NormalTok{) }\KeywordTok{as}\NormalTok{ ts\_2021}
\KeywordTok{from}\NormalTok{ yearly\_values\_long}
\KeywordTok{group} \KeywordTok{by}\NormalTok{ ts\_id}
\end{Highlighting}
\end{Shaded}

\bookmarksetup{startatroot}

\hypertarget{unpivoting}{%
\chapter{Unpivoting}\label{unpivoting}}

I wish you don't ever need to unpivot manually in SQL. A universal way
to unpivot in SQL is to take each column of interest and do a UNION ALL.

\begin{Shaded}
\begin{Highlighting}[]

\KeywordTok{select} 
\NormalTok{  ts\_id, }
  \StringTok{\textquotesingle{}2020\textquotesingle{}} \KeywordTok{as}\NormalTok{ ts\_year, }
\NormalTok{  [}\DecValTok{2020}\NormalTok{] }\KeywordTok{as}\NormalTok{ ts\_value}
\KeywordTok{from}\NormalTok{ yearly\_values\_wide}
\KeywordTok{union} \KeywordTok{all}
\KeywordTok{select} 
\NormalTok{  ts\_id, }
  \StringTok{\textquotesingle{}2021\textquotesingle{}} \KeywordTok{as}\NormalTok{ ts\_year, }
\NormalTok{  [}\DecValTok{2021}\NormalTok{] }\KeywordTok{as}\NormalTok{ ts\_value}
\KeywordTok{from}\NormalTok{ yearly\_values\_wide}

\end{Highlighting}
\end{Shaded}

\bookmarksetup{startatroot}

\hypertarget{references-1}{%
\chapter{References}\label{references-1}}

https://sqlperformance.com/2019/09/t-sql-queries/t-sql-pitfalls-pivoting-unpivoting



\end{document}
